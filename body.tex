\section{Introduction \& Motivations}




{\large \bf  TEST }

%%%%%

Depending on the co-occurrence quantification, and depending on the composition for s-v-o, for some expressions, one  gets equivalences between geometric and probabilistic (and information-theoretic) expressions.


HELLO

\cite{Wong/Yao:95}:
\[
\vec{d} \cdot \vec{q}
\]
\[
\vec{d} =
	\left(
	\begin{matrix}
	P(t_1|d)\\
	P(t_2|d)\\
	P(t_3|d)\\
	\end{matrix}
	\right),
\qquad
\vec{q} =
	\left(
	\begin{matrix}
	P(q|t_1)\\
	P(q|t_2)\\
	P(q|t_3)\\
	\end{matrix}
	\right)
\]
\[
P(q|d) = \sum_t P(q|t) \cdot P(t|d)
\]
Rewrite as follows:
\[
P(q|d) = \frac{1}{P(d)} \cdot \sum_t P(q|t) \cdot P(d|t) \cdot P(t)
\]

The reformulation helps
to make $\vec{d}$ and $\vec{q}$ having the same semantics:
\[
\vec{d} =
	\left(
	\begin{matrix}
	P(d|t_1)\\
	P(d|t_2)\\
	P(d|t_3)\\
	\end{matrix}
	\right)
\]
Generalised inner product:
\[
\vec{d}^T \cdot G \cdot \vec{q}
\]
\[
G=
\left[
\begin{matrix}
P(t_1) \\
0 & P(t_2) \\
0 & 0 & P(t_3) \\
\end{matrix}
\right]
\]


%The starting point of this approach is the  window-based  vector space models of natural language \cite{Schutze}, also referred to by \emph{distributional} models of meaning. These are inspired by  the vector space models of  Information Retrieval \cite{Salton}. In IR, terms are vectors whose coordinates are their degree of appearance in documents, and symmetrically documents can also be seen as vectors whose coordinates are the degree of appearance of words in them. The vector spaces in which meanings of words live have documents as basis vectors. These models have been used to retrieve documents that are relevant to a query, seen as a collection of terms (words). The window-based vector space models are slightly different, herein,  basis vectors are other words (sometimes also referred to by context words or features). In such models, one fixes a neighbourhood window of $k$ words (e.g. $k = 5$) then computes the degree of co-occurrence of a word with other words within that window. Hence, each words will have a vectors and the distance between the word vectors can be used to reason about their  similarity of meanings. The origins of this work and its application to word meaning, similarity, and disambiguation  goes back to the work of Schutze. Later various different normalisation schemes and distances were established and experimented with and it turned out the the cosine of the angle between the vectors is a good measure of similarity, e.g. see \cite{BullerinaLevy,Curran}. 




\section{rather obvious I suppose}

\subsection{Single Terms}

In IR, the main measure of retrieval is TF-IDF. In its most basic form, given a term $t$,  a document $d$, and a set of documents $D$, this is the frequency  of   $t$  in $d$, multiplied by  the logarithm of the  total number of documents $N$ divided by the number of documents with $t$ in them. 

\begin{equation}
\label{tf-idf}
TF(t,d) \cdot IDF(t, D) = freq(t, d) \cdot \log{{N} \over {\mid \{d \in D \mid t \in d\}\mid}}
\end{equation}

In DS, given is a corpus of text in which one works with the frequency of co-occurrence of terms with a set of  features, in a window of length $k$. Given a set of $m$ features $F = \{f_1, f_2, \cdots, f_m\}$, the space is the vector space with $F$ as its basis, that is $V = \{f_i\}_i$. The vector representation of a term $t$ is a linear combination of the basis:

\begin{equation}
\label{vector-word}
\overrightarrow{t}  = \sum_i  C_i  \overrightarrow{f}_i
\end{equation}

The coordinate $C_i$ over the basis vector $\overrightarrow{f}_i$ is a normalised function of the frequency of co-occurrence of $t$ with $f_i$ in the window of length $k$. Denoting the latter by  $g(freq_k(t,f_i))$, we obtain 

\begin{equation}
C_i = g(freq_k(t,f_i))
\end{equation}

Forgetting about the $i$ indices for a minute, note that 

\begin{equation}
freq_k(t,f_i) = {{\sum_i N(t,f_i)} \over {k}}
\end{equation}

for $N(t, f)$ the number of times $t$ and $f$ occurred  $k$ words close to each other.  So in DS, one works with the frequency of co-occurrence of terms with features rather than with documents. Hence, if one replaces the $d$'s in equation \ref{tf-idf} with $f$'s, one obtains:

\begin{equation}
\label{f-for-d}
freq_k(t, f) \cdot \log{{m} \over {\mid \{f \in F \mid N(t,f)  \neq 0 \}\mid}}
\end{equation}

\[
N(t, f) * log ( m / feature_freq ( t, F ) )
\]


$P_L ( t ) = sum_f N( t, f)  /  sum_{ t' }$

$P_L ( t | F ) = n (t , F) / N ( F )$

$ n ( t , F ) = sum_{f in F} N(t, f)$


{\bf Analogy: document  freq and / feature freq}

$N(t,f)$ and $TF(t,d)$




number of Locations term t occurs in / number of Locations

means

number of times a word occurs / length of corpora

$sum_f N(t,f)$

With the right choice of event spaces, one can establish

\begin{eqnarray}
P(t) &=&  {{\sum_f N(f,t)} \over {kL}} = {{freq_k(t,f)} \over {L}}\\
P_F (t) &=&  {{\mid \{f \in F \mid N(t,f)  \neq 0 \}\mid} \over {m}}
\end{eqnarray}
for $L$ the number of times $t$ occurred in the corpus as a whole.  Hence one can rewrite equation \ref{f-for-d} as

\begin{equation}
freq_k(t,f) \cdot \log{1 \over P(t|f)}
\end{equation}
This is the same as saying that  the normalisation function $g$ in DS is a multiplication by $ \log{1 \over P(t|f)}$:

\[
g(freq_k(t, f)) = freq_k(t,f) \cdot \log{1 \over P(t|f)}
\]
This  is indeed an IDF value in which features are seen as documents, we refer to this by `TF-IFF', where `IFF' is for Inverse Feature Frequency. Herein, one obtains a term vector as follows:

\begin{equation}
\label{vector-word-idf}
\overrightarrow{t}  = \sum_i  freq_k(t,f_i) \cdot \log{1 \over P(t|f_i)}  \overrightarrow{f}_i
\end{equation}


It is surprising however,   that this is not a familiar normalisation scheme in DS. The  most known and worked-with of the latter are  Conditional Probability: $\operatorname{P}_k(f|t)$,  Likelihood Ratio: $\operatorname{LR}_k(f,t) 
= {{P(f \mid t)} \over{P(f)}}$,  and PMI$_k$: $PMI_k (f, t) =  \log{{P(f \mid t)} \over{P(f)}}$.  One simple way to relate these to IR is to say that $g$ is the the multiplication of any of these with the same  IFF as above. 


From these,  the most relevant to IR seems to be PMI. This resembles the Document Term Independence (DTI) measure, where  for a single term $t$ we have:

\begin{equation}
\label{independence}
\textmd{DTI}(d,t) := \log\frac{P(d | t)}{P(d)} = \log\frac{P(t|d)}{P(t)}
\end{equation}


Here  ${1} \over {P(d)}$ is the IDF and  $P(d| t)$ is the TF. 





Other normalisation schemes such as conditional probability, LMI, BM25 etc ???

\subsection{Phrases}

In CDS


In IR



\begin{eqnarray}
\label{tf-idf-phrase}
\RSV_\textmd{TF-IDF}(d,q) &:=& \sum_{t \in q} \textmd{score}(t,d,q)\\
\textmd{score}(t,d,q) &:=&
\TF(t,d) \cdot \IDF(t)
\end{eqnarray}

\section{Why the independence measure is best}

Multiplication of three vectors: s, v and o.
\[
\frac{
P(c|s) \cdot P(c|v) \cdot P(c|o)}{
P(c) \cdot P(c) \cdot P(c)} =
\frac{
P(s|c) \cdot P(v|c) \cdot P(o|c)
}{
P(s) \cdot P(v) \cdot P(o)
}
\]
Let $t_i$ and $t_j$ be two propositions
(e.g.~s-v-o).

$\textmd{sim}(t_i,t_j)$ is high if
\begin{enumerate}
\item many features occur within the context of subject, verb, and object
\item the features are rare
\end{enumerate}

cos: corresponds to sum over independence measure.
Therefore, pmi is the correct model.
Corresponds to
\[
\sum_c \log(...) = \log \prod_c(...)
\]



Does not work for different
spaces for nouns and verbs?

The multiplication yields a probability
for the subj-verb-obj sequence where all
components are independent of each other.

Multiplication of two conditional probs means then what?
Effect of L2 norm?


Relationship to LM - there should be an obvious one.


\section{Decomposing Similarity}
Decompsoed additive baseline similarity
\[
\frac{cos(sbj1,sbj2) + cos(vrb1,vrb2)) + cos(obj1,obj2)}{3}
\]

Can any of the NLP sentence similarities decompose? It does not seem so, for example
\begin{eqnarray*}
cos(sbj1+verb1,sbj2+vrb2) &\neq& cos(sbj1, sbj2) + cos(vrb1,vrb2) \\
cos(sbj1\odot verb1,sbj2+vrb2) &\neq& cos(sbj1, sbj2) \odot cos(vrb1,vrb2)
\end{eqnarray*}
For kronecker composition has to move between vector of different tensor rank, i.e. either from $V$ to $V \otimes V$ or from the latter to the former. Diagonalization might be helpful here, also convolution kernel might help. 

In general, may be the sentence cosines, do doecompsoe but not in a direct way as above, that is when left and right have the same operation. The operations might vary from lef to right, or the ratios might be preserved etc. 



\section{cosines again}

\begin{eqnarray}
\cos(angle(s\odot v\odot o, ...) &=&  1/ \sum P(s,v,o|c) \cdot P(c)\\
\cos(angle(s+v+o, ...) &=& \sum (P(s|c)  \cdot  ... + P(v|c) ... + ....)  \cdot  ....\\
\cos(angle(s+v+o, ...) &=&  \sum (\log{\Pi_{x in s,v,o} (P(x,c)/P(x) \cdot P(c)) ...}\\
\end{eqnarray}

\section{various log measures}

\begin{eqnarray}
\textmd{NLogP}&   N(t,c)  \cdot \log{1}{P(c)}\\
\textmd{LogNLogP}&  log(N(t,c)+1) \cdot \log{1}{P(c)}\\
\textmd{PExpN}&    P(c)^{-N(t,c)}\\
\textmd{PExpLogN}& P(c)^{-\log{N(t,c)+1}}\\
\end{eqnarray}

\begin{table*}[htbp]
\begin{tabularx}{\textwidth}{l|XX}
& NLP & IR\\
\hline
co-occurrence
&
        co-occurrence between the semantic symbol (target word) and feature words
        &
        co-occurrence between the semantic symbols (words) themselves
        \\
        \hline
representation of words
        & distributional & symbolic\\
        \hline
single vs set & similarity between two
    \emph{single phrases}
    & relationship (implication, entailment) between two
    \emph{sets of phrases}\\

symmetric yes/no &
    similarity is a symmetric function &
relationship between sets is not symmetric; moreover, the phrase-based score is not necessarily symmetric\\
\hline
similar/relevant
& phrase $t_i$ is similar to phrase $t_j$ &
document~$d$ (source) is relevant with respect to query~$q$ (target)\\
\hline
scores
& the similarity score is estimated
based on the distance/angle between distributional vectors
& the relevance score is estimated
based on the retrieval model that computes the implication between the set of document and the set of query propositions\\
probabilistic semantics
& (in)dependence between target word and feature word: $\frac{P(w_t,w_f)}{P(w_t) \cdot P(w_f)}$
& (in)dependent between document and query:
$\frac{P(d,q)}{P(d) \cdot P(q)}$\\
\hline
& for this work: virtual query has exactly one proposition; virtual document has exactly one condition; therefore, the similarity score $\textmd{sim}(\textmd{phrase1},\textmd{phrase2})$ can be compared to the retrieval score
$\textmd{RSV}(\textmd{document: set of phrases},\textmd{query: set of phrases})$.

TODO: how to get this over two columns
\\
        \hline
\end{tabularx}
\end{table*}




%%% Local Variables:
%%% mode: latex
%%% TeX-master: "paper"
%%% End:
