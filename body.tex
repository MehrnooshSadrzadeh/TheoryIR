\section{Introduction \& Motivations}


The term TF-IDF is often used in DS as a weighting scheme when it comes to normalizing  the number of times a target term occurred close to (e.g. in a window of $k$-words) a feature. The notion of TF-IDF comes from and is a well-known notion, however, this known notion is not what is used as TF-IDF in DS. Here ,  most known and worked-with of  is point wise mutual information or  PMI, defined as follows:
\[
 PMI (f, t) =  \log{{P(f \mid t)} \over{P(f)}}
 \]
ADD REFERENCE TO MIRELLA'S PAPER AND MAY BE OTHER THINGS

In this paper first we solve the mystery of why  the above (and its brothesr and sisters: LMI, Likelihood Ration, and PPMI) may be called TF-IDF or be a TF-IDF-based measure. We discover new relationships between IR's TF-IDF and DS's PMI. It turns out that the IR notion of `relevance'  is also closely related to the DS notion of `cosine similarity'. 

Then, from single terms we move to phrases and show how DS notion of  compositionality relates to a similar notion in  IR's, used in semantic IR and applied to propositions. 


\hrulefill

Depending on the co-occurrence quantification, and depending on the composition for s-v-o, for some expressions, one  gets equivalences between geometric and probabilistic (and information-theoretic) expressions.


\cite{Wong/Yao:95}:
\[
\vec{d} \cdot \vec{q}
\]
\[
\vec{d} =
	\left(
	\begin{matrix}
	P(t_1|d)\\
	P(t_2|d)\\
	P(t_3|d)\\
	\end{matrix}
	\right),
\qquad
\vec{q} =
	\left(
	\begin{matrix}
	P(q|t_1)\\
	P(q|t_2)\\
	P(q|t_3)\\
	\end{matrix}
	\right)
\]
\[
P(q|d) = \sum_t P(q|t) \cdot P(t|d)
\]
Rewrite as follows:
\[
P(q|d) = \frac{1}{P(d)} \cdot \sum_t P(q|t) \cdot P(d|t) \cdot P(t)
\]

The reformulation helps
to make $\vec{d}$ and $\vec{q}$ having the same semantics:
\[
\vec{d} =
	\left(
	\begin{matrix}
	P(d|t_1)\\
	P(d|t_2)\\
	P(d|t_3)\\
	\end{matrix}
	\right)
\]
Generalised inner product:
\[
\vec{d}^T \cdot G \cdot \vec{q}
\]
\[
G=
\left[
\begin{matrix}
P(t_1) \\
0 & P(t_2) \\
0 & 0 & P(t_3) \\
\end{matrix}
\right]
\]


%The starting point of this approach is the  window-based  vector space models of natural language \cite{Schutze}, also referred to by \emph{distributional} models of meaning. These are inspired by  the vector space models of  Information Retrieval \cite{Salton}. In IR, terms are vectors whose coordinates are their degree of appearance in documents, and symmetrically documents can also be seen as vectors whose coordinates are the degree of appearance of words in them. The vector spaces in which meanings of words live have documents as basis vectors. These models have been used to retrieve documents that are relevant to a query, seen as a collection of terms (words). The window-based vector space models are slightly different, herein,  basis vectors are other words (sometimes also referred to by context words or features). In such models, one fixes a neighbourhood window of $k$ words (e.g. $k = 5$) then computes the degree of co-occurrence of a word with other words within that window. Hence, each words will have a vectors and the distance between the word vectors can be used to reason about their  similarity of meanings. The origins of this work and its application to word meaning, similarity, and disambiguation  goes back to the work of Schutze. Later various different normalisation schemes and distances were established and experimented with and it turned out the the cosine of the angle between the vectors is a good measure of similarity, e.g. see \cite{BullerinaLevy,Curran}. 





\section{Single Terms}

\subsection{Basic Notation}
We start with very basic concepts from IR.  Given a  single term $t$ (in DS known as a `word'),  a document $d$, and a set of documents $D = \{d_1, d_2, \cdots, d_N\}$,  one computes  the frequency  of   $t$  in $d$, multiplied by  the logarithm of the  total number of documents $n$ divided by the number of documents with  term $t$ in them (for the remainder of this section `term' stands for `single term'). The most traditional way of writing this formula is as  follows:
\begin{equation}
\label{tf-idf}
\TF(t,d) \cdot \IDF(t, D) =
	\textmd{freq}(t, d) \cdot
	\log{{N} \over {\mid \{d \in D \mid t \in d\}\mid}}
\end{equation}
To be able to relate this to PMI, we need to first turn the different expressions of the above into probabilities.
This is possible by fixing an appropriate notion of event space, we follow the location-based approach of \cite{Roelleke:Book:2013} and introduce the following notation:
\begin{itemize}
\item $L = \{l_1, l_2, \cdots, l_L\}$ a set of locations
	\footnote{$l_L$!?}
\item $N_{\textmd{Location}}$ the total number of locations, i.e. size of $L$.
\item $n_{\textmd{Location}}(x,y)$ number of locations in which event $x$ `occurs with' event $y$.
\end{itemize}

Then can define a probability $P_{\textmd{Location}}(x)$ as follows
\[
P_{\textmd{Location}}(x \mid y) = \frac{n_{\textmd{Location}}(x,y)}{N_{\textmd{Location}}(y)}
\]
 this is the conditional probability of $x$ occurring/happening given $y$.
%
Using this notation as guideline, we proceed as follows:   first turn the IDF part of the TF-IDF formula into probabilities and develop its DS dual concepts, then do the same for TF, and  put the two together.  This provides us with effectively two dual TF-IDF notions for DS.  But which ones are used in DS and which ones are good to use? In the last subsection of this section, we show how one can obtain PMI from ??? and argue why the other  one is terribly bad. 


\hrulefill Mehrnoosh: extended notation

To prepare the notation for investigating the dualities between IR and DS,
that is between different event spaces and events,
we utilise a more formal and generic notation:
\[
\begin{array}{ll}
N_D(c) := |D_c| \textmd{: number of Documents in collection/corpus $c$}\\
N_T(c) := |T_c| \textmd{: number of Terms in collection/corpus $c$}\\
N_L(c) := |L_c| \textmd{: number of Locations in collection/corpus $c$}\\
n_D(t,c) := \mid \{d \in D_c \mid t \ \textmd{occurs in} \ d\}\mid\\
n_T(d,c) := \mid \{t \in T_c \mid d \ \textmd{contains} \ t\} \mid\\
n_L(t,c) := \mid \{l \in L_c \mid t \ \textmd{occurs at} \ l\} \mid\\
n_L(d,c) := \mid \{l \in L_c \mid d \ \textmd{contains} \ l\} \mid\\
\end{array}
\]
Then, the TF-IDF weight is:
\begin{equation}
\label{tf-idf}
\TF(t,d) \cdot \IDF(t,c) =
	\textmd{freq}(t, d) \cdot
	\log{{N_D(c)} \over n_D(t,c)}
\end{equation}
The argument of the logarithm is the Document-based term probability:
\[
P_{D}(t|c) =\frac{n_D(t,c)}{N_D(c)}
\]
In traditional IR, the numerator is refered to as the document frequency:
$\df(t,c):=n_D(t,c)$.

Is is important to make explicit the difference between the Document-based
term probability, $P_D(t|c)$, and the Location-based term probability, $P_L(t|c)$:
\[
P_{L}(t|c) =\frac{n_L(t,c)}{N_L(c)}
\]
For example, a term that occurs $n_L(t,c)=1000$ locations of $N_L(c)=10^8$ locations,
may occur in $n_D(t,c)=500$ of $N_D(c)=10^6$ documents.
%
The average document length is $\avgdl(c)=N_L(c)/N_D(c)=100$, and the
average term frequency is $n_L(t,c)/N_D(c)=1000/10^6$, whereas the
elite average is $\avgtf(t,c)=n_L(t,c)/n_D(t,c)=1000/500=2$,
i.e.~the elite average is the average
over the documents in which the term occurs.
%
The elite average is a measure of ``burstiness", i.e.~$\avgtf(t,c)=n_L(t,c)$ for
an extremely bursty term (all occurrences in one document, and $\avgtf(t,c)=1$
for a non-bursty term.
%
The averages can be utlised to transform the event spaces:
\[
P_L(t|c) \cdot \avgdl(c) = \frac{n_L(t,c)}{N_D(c)} = P_D(t|c) \cdot \avgtf(t,c)
\]
This equation (referred to as the Poisson bridge, since both sides yield
the Poisson parameter, the average term frequency) is helpful to transform
$P_L(t|c)$ into a Document-based expression, and vice versa.

Other important probabilities are
the Location-based probability $P_L(t|d)$ of a term given a document, and
the Location-based probability $P_L(d|c)$ of a document, and
the Term-based probability $P_T(d|c)$ of a document (dual to the Document-based probability of a term).

$P_L(t|d)$ is also referred to as the within-document probability.
\[
P_L(t|d) = \frac{n_L(t,d)}{N_L(d)}
\]

$P_L(d|c)$ is the probability of document~$d$ based on its document length
(number of Locations in $d$).
\[
P_L(d|c) = \frac{n_L(d,c)}{N_L(c)}
\]

Instead of locations, the document probability can be based on
the distinct terms occurring in the document:
\[
P_T(d|c) = \frac{n_T(d,c)}{N_T(c)}
\]

For example, for a document $d=t_1,t_2,t_1$, the Location-based count
is $n_L(t,d)=3$ (three words), and the Term-based count is
$n_T(d,c)=2$ ($d$ contains two terms of corpus $c$). the Term-based count is

In the next section we investigate the duality options regarding the IDF.

\subsection{Dualizing IDF to lnvFeatureTermFreq and InvTargetTermFreq}
\label{subsec-IDF}


\medskip
In DS, given is a corpus of text in which one works with the frequency of co-occurrence of a \emph{target term} with \emph{feature terms},
in a window of length $k$.
%
Suppose we have a set of  $M$ feature terms  $FT = \{f_1, f_2, \cdots, f_M\}$.  To obtain an IR-like measure in DS, the first intuition that comes to mind is do the following:

\begin{enumerate}
\item   Replace the notion of a  \emph{document}  with the notion of  \emph{feature}, 
\item  Replace the notion of `a term being in a document'  with the notion of `a target  term occurring $k$ words close to a feature'.  In other words,  the  simple co-occurrence concept `within document' is replaced with the window-based co-occurrence concept of  `$k$-closed to feature'. 
\end{enumerate}

This provide  us with the notion of {\bf Inverse Feature Term Frequency}  (IFTF)  (the word `term'  is to emphasise that features are words\footnote{Features need not always be words, but for the cases which they are not, they are associated with a set words, so when it comes down to extracting numerical counts from a corpus, one is considering single words, one at a time. }),   as follows:

\begin{equation}
\label{iftf}
\textmd{IFTF}(t, FT) = \log{{M} \over {\mid \{f \in FT \mid t  \ k\textmd{close-to} \ f \} \mid}}
\end{equation}
where $ \mid \{f \in FT \mid t  \ k\textmd{close-to} \ f \} \mid$ is the total number of features to which target  term $t$ occurred $k$ words close to. 

In order to view things probabilistically, first recall the same process in IR where there we have the following assignments to the location based notation:
\[
L := D\qquad
x := t \qquad y := d
\]
\[
P_{\textmd{Location}}(t \mid c) =  \frac{n_{\textmd{Location}}(t,c)}{N_{\textmd{Location}}(c)} 
\]
\[
P_{\textmd{Location}}(t \mid d) =  \frac{n_{\textmd{Location}}(t,d)}{N_{\textmd{Location}}(d)} 
\]
Then one obtains  a $P_{\textmd{Doc}}(t)$,  as follows:
\[
P_{\textmd{Doc}}(t\mid c) = \frac{n_{\textmd{Doc}}(t,c)}{N_{\textmd{Doc}}(c)} 
\]
and the IDF expression
becomes equivalent to the following 
\footnote{is defined as follows}
\begin{equation}
\label{eqn:idf}
\IDF(t, c) := \textmd{InvDocFreq}(t,c) := \log{\frac{1}{P_{\textmd{Doc}}(t \mid c)}}
\end{equation}


\hrulefill Mehrnoosh: see here

Accordingly, there is the Term-based document probability:
\[
P_{\textmd{Term}}(d\mid c) = \frac{n_{\textmd{Term}}(d,c)}{N_{\textmd{Term}}(c)} 
\]
The Term-based document probability is high for documents that contain
many distinct terms.

This leads to the notion of the ITF:
\begin{equation}
\label{eqn:itf}
\ITF(d, c) :=
	\textmd{InvTermFreq}(d,c) := \log{\frac{1}{P_{\textmd{Term}}(d \mid c)}}
\end{equation}
Comparing eqn~\ref{eqn:idf} against eqn~\ref{eqn:itf} makes explicit the
dualities: $\textmd{Term}\equiv\textmd{Doc}$ and $t\equiv d$.

The ITF of a document is high if it contains few distinct terms, i.e.~the ITF
is high for short documents (short in the sense of few distinct terms,
the multiple occurrences of a term are not considered).

Next, we discuss the duality between IR and DS.
The question is which IR dimension corresponds to which DS dimension,
i.e.~which of target term and feature term corresponds to the document
dimension, and which corresponds to the term dimension.

\hrulefill

\subsection{IDF and InvFeatureTermFreq}

Dualizing documents to features,
we get the following assignments for the location notation:
\[
L := F \qquad
x := f\qquad
y := t
\]

\newcommand{\featureterm}{\textmd{ft}}
\newcommand{\targetterm}{\textmd{tt}}
\[
D := F \qquad
d := \featureterm\qquad
t := \targetterm
\]

And obtain a $P_{\textmd{FeatureTerm}}(\targetterm|c)$ (probability of picking a feature with target term $t$ occurring $k$-close to it) as follows
\footnote{prob of picking a feature???}
\[
P_{\textmd{Locations}}(\targetterm \mid \featureterm) = \frac{n_{\textmd{Locations}}(\targetterm,\featureterm)}{N_{\textmd{Locations}} (\featureterm)} 
\]
\[
P_{\textmd{FeatureTerm}}(\targetterm \mid c) = \frac{n_{\textmd{FeatureTerm}}(\targetterm,c)}{N_{\textmd{FeatureTerm}} (c)} 
\]
\[
P_{\textmd{TargetTerm}}(\featureterm \mid c) = \frac{n_{\textmd{TargetTerm}}(\featureterm,c)}{N_{\textmd{TargetTerm}} (c)} 
\]
where $n_{\textmd{FeatureTerm}}(t,f)$ is the number of times target term $t$ occurred $k$-close to feature $f$ and $N_F$ is the total number of features. Hence the probabilistic version of IFTF becomes as follows:
\begin{equation}
%\label{eqn:InvFeatureTermFreq}
\label{eqn:iftf}
\textmd{InvFeatureTermFreq}(\targetterm, c) :=
	\log{\frac{1}{P_{\textmd{FeatureTerm}}(\targetterm \mid c)}}
\end{equation}

\medskip
The above computations provide us with a vector representation for documents in IR and for features in DS. However, in DS, one is interested in having vectors for target terms.

These are known in DS as target words and  are words in whose meanings and the similarity of meanings therein we are interested.
This option is the DS dual of IR's  InvTermFreq(doc) used in the context of machine translation, for which one can define a $P_{\textmd{TargetTerm}}(f) $ as the  probability of picking a target term  occurring $k$-close to feature $f$: 

\subsection{IDF and InvTargetTermFreq}
\[
P_{\textmd{TargetTerm}}(\featureterm|c) =
\frac{n_{\textmd{TargetTerm}}(\featureterm, c)}{N_{\textmd{TargetTerm}}(c)}
\]
\footnote{changed $f$ to $c$}
where  $N_{\textmd{TargetTerm}}$ is the total number of target terms,    $n_{\textmd{TargetTerm}}(f, t)$is the number of  times feature term $f$ occurred   $k$-close to target term  $t$,  and TT is the total number of target terms. We thus obtain the notion of  {\bf Term Frequency-Inverse Target Term Frequency} (ITTF) as follows:

\begin{equation}
%\label{eqn:InvTargetTermFreq}
\label{eqn:ittf}
\textmd{InvTargetTermFreq}(\featureterm, c) :=
\log{\frac{1}{P_{\textmd{TargetTerm}}(\featureterm \mid c)}}
\end{equation}
Comparing eq~\ref{eqn:ittf} to eq~\ref{eqn:iftf} underlines the symmetry.

From an IR perspective, this notion may be  the  lesser intuitive of the two, since a direct comparison of it  with the TF-IDF formula amounts to the fact that  we replace the notion of `document' with `target terms' and the notion of `term' with `feature terms'.

\subsection{Dualizing TF to Feature Term and Target Term Frequency}
\label{subsec-TF}

TF can also be dualized from documents to features and to target terms to obtain the corresponding expressions in each computation formula.  We refer to these by Feature Term Frequency and Target Term  Frequency. We still keep the  word `term'  in these expressions and add adjectives `feature' and `target'  to it so we can distinguish them from the previous notion of term in IR. 

In IR, TF stands for the `within document term frequency'. In a location notation we have:
\[
\textmd{freq}(t,d) = n_{\textmd{Location}}(t,d)
\]
Dualizing this to features, we obtains `$k$-close to feature, target  term frequency', abbreviated to {\bf TargetTermFrequency} (TTF), as follows:
\begin{equation}
\label{t-f}
\textmd{freq}(t,f) = n_{\textmd{Location}}(t,f)
\end{equation}
This is the number of times target  term $t$ occurrs $k$-close to feature $f$.  Using reasoning from previous subsection, we know that one can also have a notion of `$k$-close to target term, feature frequency', abbreviated dually to {\bf FeatureTermFrequency} (FTF), as follows:
\begin{equation}
\label{f-t}
\textmd{freq}(f,t) = n_{\textmd{Location}}(f,t)
\end{equation}
This is the number of times a feature $f$ occurs $k$-close to a target term $t$.  This is not what one expects the DS method to be counting. Normally, DS methods count how many times a target term occurs close to a feature, i.e. the TTF. 

In an ideal DS scenario, all the words from the vocabulary of a language can be both features and targets. In this case, evidently, equations \ref{t-f} and \ref{f-t} amount to the same thing. The number of times a target term occurred $k$-close to a feature is the same as the number of times a feature occurs $k$-close to a target term, that is
\begin{equation}
\label{ttf=ftf}
\textmd{freq}(t,f) = \textmd{freq}(f,t) 
\end{equation}
However, in practical scenarios, both of these are separate subsets of the vocabulary, with feature words being specific lemmas, topics, grammatical relations, etc. 

\subsection{Dualizing TF-IDF}
\label{subsec-TF-IDF}

Putting  the equation developed in subsections \ref{subsec-IDF} and \ref{subsec-TF} together,  we obtain four dual notions  for TF-IDF in DS. If one takes the  ideal scenario to be true, then  $\textmd{freq}(t,f) = \textmd{freq}(f,t)$ and these four reduce to  two.  To be systematic we list all four options below: 

\begin{enumerate}
\item {\bf Feature Term Frequency-Inv Feature Term Frequency} (FTF.IFTF) 
\begin{equation}
\label{ftf-iftf}
\textmd{FTF}(t,f) \cdot \textmd{IFTF}(t, F) =  \textmd{freq}(f, t) \cdot \log{\frac{1}{P_{\textmd{FeatureTerm}}(t)}}
\end{equation}
\item  {\bf Target Term Frequency-Inv Feature Term Frequency} (TTF.IFTF)
\begin{equation}
\label{ttf-iftf}
\textmd{TTF}(t,f) \cdot \textmd{IFTF}(t, F) =  \textmd{freq}(t,f) \cdot \log{\frac{1}{P_{\textmd{FeatureTerm}}(t)}}
\end{equation}
\item {\bf Feature Term Frequency-Inv Target Term Frequency} (FTF.ITTF)
\begin{equation}
\label{ftf-ittf}
\textmd{FTF}(t,f) \cdot \textmd{ITTF}(t, TT) =  \textmd{freq}(f, t) \cdot \log{\frac{1}{P_{\textmd{TargetTerm}}(f)}}
\end{equation}
\item  {\bf Target Term Frequency-Inv Target Term Frequency} (TTF.ITTF)
\begin{equation}
\label{ttf-ittf}
\textmd{TTF}(t,f) \cdot \textmd{ITTF}(t, TT) =  \textmd{freq}(t, f) \cdot \log{\frac{1}{P_{\textmd{TargetTerm}}(f)}}
\end{equation}
\end{enumerate}

In the ideal case we have TTF = FTF, and hence 
\begin{quote}
FTF.IFTF = TTF.IFTF \qquad FTF.ITTF = TTF.ITTF. 
\end{quote}

The question is  which one of these two (or four)  is the one used in DS? Some DS measures (from \cite{BullineriaLevy} CITE and CHECK Turney-Pantel) and their IR duals are shown in table \ref{tb-ds-ir}, obtained by following the latter approach (TTF.ITTF) and  replacing  `document' with `target term'  ($d$ with $t$) and  `term' with `feature term'  ($t$ with $f$).
  
 
\begin{figure*}[htb]
  \centering
  \begin{tabular}{|l|c|c|l|}
  \hline
  DS Measure  && IR Dual &\\
  \hline
 Raw Counts  &$n_{\textmd{TargetTerm}}(f,t)$ & TF &$n_{\textmd{Doc}}(t,d)$\\
 &&&\\
 Conditional Probability &$ \frac{P(f,t)}{P(t)}$ & unknown &$\frac{P(t,d)}{P(d)}$\\
 &&&\\
 Ratio Likelihood & $\frac{P(f \mid t)}{P(f)P(t)}$ & unknown & $\frac{P(t \mid d)}{P(t)P(d)}$\\
 &&&\\
 PMI & $\log{\frac{P(f \mid t)}{P(f)P(t)}}$  & see  subsection \ref{PMI-TF-IDF}&$\log{\frac{P(t \mid d)}{P(t)P(d)}}$\\
 \hline
  \end{tabular}
  \caption{Average score (Pearson) correlation per query.}
  \label{tb-ds-ir}
\end{figure*}

  In the next section we show that how the lesser intuitive notion of  {\bf TTF-ITTF} can be used to get DS's {\bf PMI}. 

\subsection{PMI and FTF-ITTF}
\label{PMI-TF-IDF}

The PMI is a favourable feature value for the vector-based representation
of a term.
The PMI is also the basis of TF-IDF, and we show-case the relationship
in the equations to follow.

Let $d$ be a document, $q$ a query, and $t$ a term.
The logarithm of likelihood ratio of document and query is:
\[
\log\frac{P(d|q)}{P(d)} =
	\sum_t n_L(t,d) \cdot \log\frac{P(t|q)}{P(t)}
\]
Here, $n_L(t,d)=\textmd{freq}(t,d)$ is the within-document term frequency
quantification.

The likelihood ratio is a measure of (in)dependence.
The decomposition on the right side leads to the measure of independence
between term and query.
\[
\frac{P(t|q)}{P(t)} = \frac{P(t,q)}{P(t) \cdot P(q)} = \frac{P(q|t)}{P(q)}
\]
For NLP, let $t_i$ and $t_j$ be two target terms
(for which we want to establish a similarity).
$t_i$ is the ``document" and ``$t_j$" is the query.
Then:
\[
\log\frac{P(t_i|t_j)}{P(t_i)} =
	\sum_f \textmd{freq}(f,t_i) \cdot \log\frac{P(f|t_j)}{P(f)}
\]
The decomposition on the right side ...:
\[
\frac{P(f|t_j)}{P(f)} =
\frac{P(\textmd{feature}|\textmd{target})}{P(\textmd{feature})} =
\frac{P(\textmd{target}|\textmd{feature})}{P(\textmd{target})}
\]

The problem is that the term probability (the feature probability)
is based on the ``total" occurrence, whereas the IDF is based
on counting documents.
In other words,
the feature probability (for IR, term probability) is based on
the occurrence of the event:
$P(\textmd{feature term}):=P_{\textmd{Location}}(\textmd{feature term})$
is based on the Locations (the number of times the feature occurs), so is
$P(t):=P_{\textmd{Location}}(t)$.

Transformation to the ``document" (the target term) via Poisson bridge.
...

For IR:
\[
P(t) = P(t|q) \cdot P_{\textmd{Doc}}(t)
\]
\[
P(t|q) = \avgtf(t) / \avgdl
\]
Therefore:
\[
\log\frac{P(t|q)}{P(t)} = \log\frac{1}{P_{\textmd{Doc}}(t)}
\]

For NLP/PMI:
\[
P(f) = P(f|t) \cdot P_\textmd{TargetTerms}(f)
\]
\[
\log\frac{P(f|t_j)}{P(f)} = \log\frac{1}{P_\textmd{TargetTerms}(f)}
\]
Expresses how discriminative the feature is.


FORM INNER PRODUCT OF TWO TARGET TERMS


\section{Phrases}

Title: Phrase-based


In CDS


In IR


\subsection{XF-IDF: Scoring Subject-Verb-Object Phrases}

The main idea of XF-IDF \cite{Azzam/etal:SIGMOD:KEYS:2010} is to extend
TF-IDF towards a ranking model that supports phrases.
%
A document is retrieved if it contains at least one of the query phrases.
%
The impact of a phrase on the retrieval score is given by the XF-IDF
weight of a phrase.
%
Let $(s,v,o)$ be a subj-verb-obj phrase.
In more general, this can be a phrase of any length with any structure
(e.g.~subject, relationship, object), for the for purpose of this paper,
we focus the discussion on s-v-o phrases.
%
Let $\textmd{type}(x,(s,v,o))$ be a function that retuns the type of the component
of a phrase, i.e.~$\textmd{type}(s,(s,v,o))=\textmd{subject}$.

Then:
\[
w_\textmd{XF-IDF}((s,v,o)) :=
\sum_{x \in \{s,v,o\}} \TF(x,d) \cdot \IDF(x) \cdot P(\textmd{type}(x,(s,v,o)))
\]
To illustrate, consider the XF-IDF weight of the phrase ``woman likes sport".

...
DO EXAMPLE


WHAT IS P(type)?

\section{Relevance vs Similarity}

\subsection{Relevance}

\subsection{Similarity}

\section{Why the independence measure is best}

Multiplication of three vectors: s, v and o.
\[
\frac{
P(c|s) \cdot P(c|v) \cdot P(c|o)}{
P(c) \cdot P(c) \cdot P(c)} =
\frac{
P(s|c) \cdot P(v|c) \cdot P(o|c)
}{
P(s) \cdot P(v) \cdot P(o)
}
\]
Let $t_i$ and $t_j$ be two propositions
(e.g.~s-v-o).

$\textmd{sim}(t_i,t_j)$ is high if
\begin{enumerate}
\item many features occur within the context of subject, verb, and object
\item the features are rare
\end{enumerate}

cos: corresponds to sum over independence measure.
Therefore, pmi is the correct model.
Corresponds to
\[
\sum_c \log(...) = \log \prod_c(...)
\]



Does not work for different
spaces for nouns and verbs?

The multiplication yields a probability
for the subj-verb-obj sequence where all
components are independent of each other.

Multiplication of two conditional probs means then what?
Effect of L2 norm?


Relationship to LM - there should be an obvious one.


\section{Decomposing Similarity}
Decompsoed additive baseline similarity
\[
\frac{cos(sbj1,sbj2) + cos(vrb1,vrb2)) + cos(obj1,obj2)}{3}
\]

Can any of the NLP sentence similarities decompose? It does not seem so, for example
\begin{eqnarray*}
cos(sbj1+verb1,sbj2+vrb2) &\neq& cos(sbj1, sbj2) + cos(vrb1,vrb2) \\
cos(sbj1\odot verb1,sbj2+vrb2) &\neq& cos(sbj1, sbj2) \odot cos(vrb1,vrb2)
\end{eqnarray*}
For kronecker composition has to move between vector of different tensor rank, i.e. either from $V$ to $V \otimes V$ or from the latter to the former. Diagonalization might be helpful here, also convolution kernel might help. 

In general, may be the sentence cosines, do doecompsoe but not in a direct way as above, that is when left and right have the same operation. The operations might vary from lef to right, or the ratios might be preserved etc. 



\section{cosines again}

\begin{eqnarray}
\cos(angle(s\odot v\odot o, ...) &=&  1/ \sum P(s,v,o|c) \cdot P(c)\\
\cos(angle(s+v+o, ...) &=& \sum (P(s|c)  \cdot  ... + P(v|c) ... + ....)  \cdot  ....\\
\cos(angle(s+v+o, ...) &=&  \sum (\log{\Pi_{x in s,v,o} (P(x,c)/P(x) \cdot P(c)) ...}\\
\end{eqnarray}

\section{various log measures}

\begin{eqnarray}
\textmd{NLogP}&   N(t,c)  \cdot \log{1}{P(c)}\\
\textmd{LogNLogP}&  log(N(t,c)+1) \cdot \log{1}{P(c)}\\
\textmd{PExpN}&    P(c)^{-N(t,c)}\\
\textmd{PExpLogN}& P(c)^{-\log{N(t,c)+1}}\\
\end{eqnarray}

\begin{table*}[htbp]
\begin{tabularx}{\textwidth}{l|XX}
& NLP & IR\\
\hline
co-occurrence
&
        co-occurrence between the semantic symbol (target word) and feature words
        &
        co-occurrence between the semantic symbols (words) themselves
        \\
        \hline
representation of words
        & distributional & symbolic\\
        \hline
single vs set & similarity between two
    \emph{single phrases}
    & relationship (implication, entailment) between two
    \emph{sets of phrases}\\

symmetric yes/no &
    similarity is a symmetric function &
relationship between sets is not symmetric; moreover, the phrase-based score is not necessarily symmetric\\
\hline
similar/relevant
& phrase $t_i$ is similar to phrase $t_j$ &
document~$d$ (source) is relevant with respect to query~$q$ (target)\\
\hline
scores
& the similarity score is estimated
based on the distance/angle between distributional vectors
& the relevance score is estimated
based on the retrieval model that computes the implication between the set of document and the set of query propositions\\
probabilistic semantics
& (in)dependence between target word and feature word: $\frac{P(w_t,w_f)}{P(w_t) \cdot P(w_f)}$
& (in)dependent between document and query:
$\frac{P(d,q)}{P(d) \cdot P(q)}$\\
\hline
& for this work: virtual query has exactly one proposition; virtual document has exactly one condition; therefore, the similarity score $\textmd{sim}(\textmd{phrase1},\textmd{phrase2})$ can be compared to the retrieval score
$\textmd{RSV}(\textmd{document: set of phrases},\textmd{query: set of phrases})$.

TODO: how to get this over two columns
\\
        \hline
\end{tabularx}
\end{table*}




%%% Local Variables:
%%% mode: latex
%%% TeX-master: "paper"
%%% End:
